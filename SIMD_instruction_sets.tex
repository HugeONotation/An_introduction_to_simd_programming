\chapter{SIMD Instruction Sets \& Microarchitextures}
When utilizing SIMD instructions, it's important to be aware of the different
extensions . An important aspect of utilizing different. This chapter will
contain a brief summary and history of some contemporary SIMD instruction sets. 

\section{x86}

\textbf{MMX (1997)} x86 gained its first SIMD instruction with the MMX
extension. This instruction set utilized the ST registers introduced by the x87
extension. (The x87 instruction set was the first attempt to add instructions
for processing floating-point numbers to x86.) MMX offered 64-bit wide SIMD,
offering a small set of instructions for manipulating 8, 16, and 32-bit
integers. By modern standards, the MMX instruction set is considered outdated
and its use is discouraged. For that reason is not discussed further. It is only
mentioned to point out its irrelevance.

\subsection{SSE Family}
\textbf{SSE (1999)} This extension introduced x86's xmm registers, which are 128
bits in size. It offers instructions primarily meant for processing 32-bit
floats, as well as a smaller number of instructions meant for 8 and 16-bit
unsigned integers. SSE is considered part of the base x86-64 instruction set so
all machines which target the 64-bit variant of x86 has access to these
instructions. The 32-bit float facilities this extension offers are considered
as superseding the corresponding x87 facilities. When compiling for a program
which utilizes 32-bit floats for a modern x86 CPU, compilers will make use of
SSE instructions.

\textbf{SSE2 (2000)} This extension introduces instruction for processing 64-bit
floats, as well as 32 and 64-bit integers, as well as . Like SSE, SSE2 is part
of the base x86-64 instruction set, and SSE2 instructions are commonly emitted
when processing 64-bit floats.

\textbf{SSE3 (2004)} This is a relatively minor ISA extension. It introduces a
few operations for performing alternating additions/subtractions and
additions/subtractions on adjacent vector elements, all on floats, and a few
shuffling instructions.

\textbf{SSSE3 (2006)} This is also a relatively minor addition, with the
exception of the pshufb instruction which is broadly considered to be one of the
most powerful broadly available SIMD instructions. 

\textbf{SSE4.1 (2007)} This extension rounds out x86's .

\textbf{SSE4.2 (2007)} This extension introduces instructions meant for
processing strings and for performing cyclic redundancy checks. The instructions
meant for processing strings do not generally perform well, and may be bested by
the use of instructions for processing 8-bit elements.

\subsection{AVX Family}

\textbf{AVX (2011)} This extensions brings x86 its ymm registers, 256 bits in
size. Additionally, it brings a new encoding scheme for SIMD instructions. This
adds instructions for processing 32 and 64-bit floats. It also adds certain
instructions that improve overall quality of life, such as broadcast instructions.

\textbf{AVX2 (2013)} This instruction set generally extends the 128-bit
facilities for processing 8, 16, 32, and 64 bit integers to 256 bits.

\textbf{FMA a.k.a FMA3 (2014)} This instruction set mainly adds .

\subsection{AVX512 Family}
\textbf{AVX512F (2015)} 

\textbf{AVX512BW (2015)}

\textbf{AVX512DQ (2015)}

\textbf{AVX512VL (2015)}

\textbf{AVX512CD (2015)}

\textbf{AVX512VPOPCNTDQ (2017)}

\textbf{AVX512IFMA (2018)}

\textbf{AVX512VBMI (2018)}

\textbf{AVX512VNNI (2019)}

\textbf{AVX512VBMI2 (2019)}

\textbf{AVX512BITALG (2019)}

\textbf{AVX512BF16 (2020)}

\textbf{AVX512FP16 (2021)}

\subsection{AVX10 Family}

\textbf{AVX10.1 (Expected 2024)} This extension seeks to unify the disparate
feature set into a different. 

\textbf{AVX10.2 (Expected 2024)}

\subsection{Other Important Extensions}
\textbf{VPCLMULQDQ (2019)} 

\textbf{GFNI (2019)} 

\textbf{VAES (2019)} 

\section{ARM}
\subsection{Neon (2005)}

\subsection{SVE (2018)}

\subsection{SVE2 (2020)}

\section{Intel x86 Microarchitectures}
\section{AMD x86 Microarchitectures}
\section{ARM Holding's CPUs}
\section{Apple's CPUs}
